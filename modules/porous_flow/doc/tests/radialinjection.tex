\documentclass[]{report}
\usepackage{amsmath,amsfonts,graphicx}

\def\species{\mathrm{sp}}
\def\phase{\mathrm{ph}}
\def\massfrac{\chi}
\def\flux{\mathbf{F}}
\def\darcyvel{\mathbf{v}}
\def\energydens{\mathcal{E}}
\def\d{\mathrm{d}}

\newcommand{\uo}{\mbox{UO\textsubscript{2}}\xspace}

\setcounter{secnumdepth}{3}
\DeclareMathOperator{\erfc}{erfc}

\begin{document}


\title{Flow from a source in a 1D radial model}
\author{CSIRO}
\maketitle

\tableofcontents

\chapter{Similarity solution}

Flow from a source in a radial 1D problem admits a similarity solution that is valid for a
broad range of problems, including multi-component, multi-phase flow. Properties such as fluid
pressure, saturation and component mass fraction can all be characterised by the similarity
variable $\zeta = r^2/t$, where $r$ is the radial distance from the source, and $t$ is time.

This similarity solution can be used to verify simple 1D radial models of increasing complexity.

\chapter{Two phase immiscible flow}

The simplest test case that features a similarity solution is injection of a gas phase into a fully liquid
saturated model at a constant rate. Figure \ref{fig:theis_similarity_fig} shows the comparison of
similarity solutions calculated with either fixed radial distance or fixed time. In this case, good agreement
is observed between the two results for both liquid pressure and gas saturation. This example is included
in the automatic test suite.

\begin{figure}[htb]
\centering
\includegraphics[width=\textwidth]{theis_similarity_fig.pdf}
\caption{Comparison of similarity solutions. (a) Liquid pressure; (b) Gas saturation}
\label{fig:theis_similarity_fig}
\end{figure}

\end{document}
